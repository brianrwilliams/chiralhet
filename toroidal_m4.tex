\documentclass[12pt]{amsart}

\usepackage{slashed}
\linespread{1.25}

\usepackage{amsrefs, mathrsfs,amsmath, amscd, amsthm,amssymb, amsfonts, verbatim,subfigure, enumerate,stmaryrd}
%\usepackage[mathcal]{eucal}
\usepackage[mathscr]{euscript}
\usepackage[all,cmtip]{xy}
\usepackage{graphicx}
\usepackage[top=1in, bottom=1.25in, left=1.3in, right=1.3in]{geometry}
%\usepackage{fullpage}
%\usepackage{epstopdf}
%\DeclareGraphicsRule{.tif}{png}{.png}{`convert #1 `basename #1 .tif`.png}
%Use Palatino font

\usepackage{mathpazo}
\usepackage[colorlinks=true,linkcolor=blue,citecolor=red]{hyperref}


%formatting
%\usepackage[margin=1in]{geometry}
%\geometry{letterpaper}

%%xymatrix
\usepackage[all,cmtip]{xy}
\newcommand{\xym}{\xymatrix@1@=14pt@M=2pt}


\usepackage{tikz}
\usepackage{tikz-cd}
\usetikzlibrary{arrows,shapes}
\usetikzlibrary{trees}
\usetikzlibrary{matrix,arrows}
\usetikzlibrary{positioning}
\usetikzlibrary{calc,through}
\usetikzlibrary{decorations.pathreplacing}
\usepackage{pgffor}
%\usepackage{tikz-feynman} 

\usetikzlibrary{decorations.pathmorphing}
\usetikzlibrary{decorations.markings}
\tikzset{
	% >=stealth', %%  Uncomment for more conventional arrows
    vector/.style={decorate, decoration={snake}, draw},
	provector/.style={decorate, decoration={snake,amplitude=2.5pt}, draw},
	antivector/.style={decorate, decoration={snake,amplitude=-2.5pt}, draw},
    fermion/.style={draw=black, postaction={decorate},
        decoration={markings,mark=at position .55 with {\arrow[draw=black]{>}}}},
    fermionbar/.style={draw=black, postaction={decorate},
        decoration={markings,mark=at position .55 with {\arrow[draw=black]{<}}}},
    fermionnoarrow/.style={draw=black},
    gluon/.style={decorate, draw=black,
        decoration={coil,amplitude=4pt, segment length=5pt}},
    scalar/.style={dashed,draw=black, postaction={decorate},
        decoration={markings,mark=at position .55 with {\arrow[draw=black]{>}}}},
    scalarbar/.style={dashed,draw=black, postaction={decorate},
        dwecoration={markings,mark=at position .55 with {\arrow[draw=black]{<}}}},
    scalarnoarrow/.style={dashed,draw=black},
    electron/.style={draw=black, postaction={decorate},
        decoration={markings,mark=at position .55 with {\arrow[draw=black]{>}}}},
	bigvector/.style={decorate, decoration={snake,amplitude=4pt}, draw},
}

\pagestyle{plain}
\newtheorem{theorem}{Theorem}[section]
\newtheorem{prop}[theorem]{Proposition}
\newtheorem{lemma}[theorem]{Lemma}
\newtheorem{cor}[theorem]{Corollary}
\newtheorem{claim}[theorem]{Claim}

\theoremstyle{definition}
\newtheorem{dfn}[theorem]{Definition}
\newtheorem{dfn/lem}{Definition/Lemma}
\newtheorem{lesson}[theorem]{Lesson}

\theoremstyle{remark}
\newtheorem{rmk}[theorem]{Remark}
\newtheorem{eg}[theorem]{Example}
\newtheorem{construction}[theorem]{Construction}


\linespread{1.25}

\usepackage{parskip}
\setlength{\parindent}{18pt}
\setlength{\parindent}{0cm}


\usepackage{color}   %May be necessary if you want to color links
\usepackage{hyperref}
\hypersetup{
    colorlinks=true, %set true if you want colored links
    linktoc=all,     %set to all if you want both sections and subsections linked
    linkcolor=blue,  %choose some color if you want links to stand out
}


%\numberwithin{equation}{section}
%\numberwithin{example}{section}
%\numberwithin{definition}{section}


%%%%%%%%%%%%%%%%%%%%%%         Defintions        %%%%%%%%%%%%%%%%%%%%%%%%%%%%%%%%%%
\newcommand{\on}{\operatorname}
\newcommand{\ol}{\overline}
\newcommand{\nc}{\newcommand}
\nc{\wt}{\widetilde}

%%%%%%%%%%%%%%%%%%%%%%%% Symbols %%%%%%%%%%%%%%%%%%%


%\newcommand{\CC}{\mathbb{C}}
\newcommand{\N}{\mathbb{N}}
\newcommand{\R}{R}
\nc{\cR}{\mathcal{R}}
\newcommand{\Q}{\mathbb{Q}}
\newcommand{\Z}{\mathbb{Z}}
\newcommand{\Etau}{{\text{E}_\tau}}
\newcommand{\E}{{\mathcal E}}
\nc{\U}{\mathbf{U}}
%\newcommand{\F}{\mathbf{F}}
%\newcommand{\G}{\mathbf{G}}
\newcommand{\eps}{\epsilon}
%\newcommand{\g}{\mathbf{g}}
\newcommand{\im}{\op{im}}
%%%%%%%%%%%%%%%%%%%%%%         Functions         %%%%%%%%%%%%%%%%%%%%%%%%%%%%%%%%%%%
\providecommand{\abs}[1]{\left\lvert#1\right\rvert}
\providecommand{\norm}[1]{\left\lVert#1\right\rVert}
\newcommand{\abracket}[1]{\left\langle#1\right\rangle}
\newcommand{\bbracket}[1]{\left[#1\right]}
\newcommand{\fbracket}[1]{\left\{#1\right\}}
\newcommand{\bracket}[1]{\left(#1\right)}
\providecommand{\from}{\leftarrow}
\newcommand{\bl}{\textbf}
\newcommand{\mbf}{\mathbf}
\newcommand{\mbb}{\mathbb}
\newcommand{\mf}{\mathfrak}
\newcommand{\mc}{\mathcal}
\newcommand{\cinfty}{C^{\infty}}
\newcommand{\pa}{\partial}
\newcommand{\prm}{\prime}
%\renewcommand{\dbar}{\bar\pa}
\newcommand{\OO}{{\mathcal O}}
\newcommand{\hotimes}{\hat\otimes}
\newcommand{\BV}{Batalin-Vilkovisky }
\newcommand{\CE}{Chevalley-Eilenberg }
\newcommand{\suml}{\sum\limits}
\newcommand{\prodl}{\prod\limits}
\newcommand{\into}{\hookrightarrow}
\newcommand{\Ol}{\mathcal O_{loc}}
\newcommand{\mD}{{\mathcal D}}
\newcommand{\iso}{\cong}
\newcommand{\dpa}[1]{{\pa\over \pa #1}}
\newcommand{\PP}{\mathrm{P}}
\newcommand{\Kahler}{K\"{a}hler }
\newcommand{\Fock}{{\mathcal Fock}}



\nc{\CEcoh}{\mathcal{C}^{*}}


%%%%%%%%%%%%% Lie algebras
\nc{\fh}{\mathfrak{h}}
\nc{\fg}{\mathfrak{g}}
\nc{\fghat}{\widehat{\fg}}
\nc{\fn}{\mathfrak{n}}
\nc{\cF}{\mc{F}}
\nc{\CC}{\mathbb{C}}
\nc{\Linf}{L_{\infty}}
\nc{\cL}{\mc{L}}



\nc{\delbar}{\overline{\partial}}
\nc{\del}{\partial}
\nc{\dd}{d}

\nc{\vac}{|0\rangle}

\nc{\cK}{\mc{K}}
\nc{\opqm}[2]{\Omega^{#1,#2}_{m}}
\nc{\fgtil}{\wt{\fg}}
%\nc{\CC}{\mathcal{C}_{*}}
\nc{\Sym}{\on{Sym}}
\nc{\dzbar}{d \overline{z}}

\nc{\symcat}{\on{\bf C}^{\otimes}}
\nc{\Fcal}{\mathcal{F}}
\nc{\sF}{\mc{F}}

\nc{\Vect}{\on{Vect}}
\nc{\dgVect}{\on{dg-Vect}}

\nc{\ip}{\langle \bullet , \bullet \rangle}
\nc{\ses}[3]{0 \rightarrow #1 \rightarrow #2 \rightarrow #3 \rightarrow 0}


\renewcommand{\Im}{\op{Im}}
\renewcommand{\Re}{\op{Re}}
%%%%%%%%%%%%%%%%%%%%%%     Math    Operators         %%%%%%%%%%%%%%%%%%%%%%%%%%%%%%%
\DeclareMathOperator{\mHom}{\mathcal{H}om}
\DeclareMathOperator{\End}{End}
\DeclareMathOperator{\Supp}{Supp}
%\DeclareMathOperator{\Sym}{Sym}
\DeclareMathOperator{\Hom}{Hom}
\DeclareMathOperator{\Spec}{Spec}
\DeclareMathOperator{\Deg}{Deg}
\DeclareMathOperator{\Diff}{Diff}
\DeclareMathOperator{\Ber}{Ber}
\DeclareMathOperator{\Vol}{Vol}
\DeclareMathOperator{\Tr}{Tr}
\DeclareMathOperator{\Or}{Or}
\DeclareMathOperator{\Ker}{Ker}
\DeclareMathOperator{\Mat}{Mat}
\DeclareMathOperator{\Ob}{Ob}
\DeclareMathOperator{\Isom}{Isom}
\DeclareMathOperator{\PV}{PV}
\DeclareMathOperator{\Der}{Der}
\DeclareMathOperator{\HW}{HW}
\DeclareMathOperator{\Eu}{Eu}
\DeclareMathOperator{\HH}{H}
\DeclareMathOperator{\Jac}{Jac}
\DeclareMathOperator{\Res}{Res}

\def\d{{\rm d}}
\def\tensor{\otimes}
\def\Hat{\widehat}
\def\xto{\xrightarrow}


\def\brian{\textcolor{blue}{BW: }\textcolor{blue}}
\def\matt{\textcolor{red}{MS:}\textcolor{red}}


%%%%%%%%%%%%%%%%%%%%%%%%%%%%%% Allow display breaks within equations %%%%%%%%%%%%%%%%%%%%
\allowdisplaybreaks[4]  %%%%%%%%%%% choose 1-4, where 4 is the strongest desire to breack

\begin{document}

 \title{Factorization and vertex algebras from  holomorphic fibrations}
  \author{Matt Szczesny, Jackson Walters, and Brian Williams}
  \date{}

  
  \maketitle

\begin{abstract}
Let $X$ be a complex manifold, $\pi: Y \rightarrow X$ a locally trivial holomorphic fibration with fiber $F$, and $(\fg, \ip )$ a Lie algebra with an invariant symmetric form. We show how to associate to this data a holomorphic factorization algebra  $\cF_{Y/X}(\fg)$ on $X$ in the formalism of Costello-Gwilliam via a type of pushforward operation. When $X=\mathbb{C}$, this construction produces a vertex algebra which is a vacuum module for the universal central extension of $\fg \otimes H^{0}(F, \mc{O})[z,z^{-1}]$. In particular,  when $F$ is a torus $(\CC^{*})^n$, we recover a vertex algebra naturally associated to an $n+1$--toroidal algebra. We give an explicit description of the chiral homology of $\cF_{Y/X}(\fg)$.
\end{abstract}


\tableofcontents


\section{Introduction}

Suppose $\fg$ is a Lie algebra. 
By definition, a universal central extension of $\fg$ is a central extension of $\fg$ such that all other central extensions are pulled back from it. 
Universal central extensions are unique if they exist. 
As an example, consider the (infinite dimensional) Lie algebra $\fg[z,z^{-1}] = \fg \tensor \mathbb{C}[z,z^{-1}]$ where $\fg$ is a complex simple Lie algebra. 
Here, $\mathbb{C}[z,z^{-1}]$ is the commutative algebra of Laurent polynomials and the resulting Lie algebra $\fg[z,z^{-1}]$ is known as the {\em loop algebra} of $\fg$. . 
In \cite{Garland} \brian{is this reference correct?} it is shown that there exists a universal central extension of $\fg[z,z^{-1}]$ of the form
\[
\mathbb{C} \to \Hat{\fg} \to \fg[z,z^{-1}] .
\]
These extensions are known as {\em affine algebras} and are \brian{...}.

In \cite{Kassel}, a generalization of this universal central extension is considered where $\mathbb{C}[z,z^{-1}]$ is replaced by an arbitrary commutative ring $R$. 
That is, one considers the Lie algebra $\fg_R = \fg \tensor R$ where the Lie bracket is defined by 
\[
[X \tensor r, Y \tensor s] = [X,Y] \tensor rs .
\]
It is shown that there exists a universal central extension of the form
\[
H_2^{\rm Lie}(\fg_R) \to \Hat{\fg}_R \to \fg_R .
\]
Furthermore, when $\fg$ is simple there is an isomorphism of the Lie algebra homology $H_2(\fg_R) \cong \Omega^1_R / \d R$ where $\Omega^1_R$ is the $R$-module of K\"{a}hler differentials and the quotient is by all exact differentials. 
We will review the precise form of the cocycle defining this central extension below. 

In this paper, we start with the data of a finite dimensional Lie algebra $\fg$ equipped with an invariant symmetric form and a holomorphic fibration $\pi : Y \to X$. 
To such data we will associate an extension of local Lie algebras \brian{??}.
In the case that $Y = X$ is a Riemann surface and $\pi$ is the identity, the extension of local Lie algebras is modest enhancement of the usual affine algebra.


\section{Lie algebras and vertex algebras}
\subsection{Lie algebras, $\Linf$ algebras, and central extensions}


Given a complex Lie algebra $\fg$ with invariant bilinear form $\ip$, and a $\CC$-algebra $R$, $ \fg_{R} := \fg \otimes_{\CC} R$ carries a natural Lie algebra structure with bracket
\[
[X \otimes r, Y \otimes s] = [X,Y] \otimes rs.
\]
By a result of Kassel ~\cite{Kassel}, the universal central extension $\fghat_{R}$ of $\fg_{R}$ can be described as follows. As an $R$-module, it is given by
\begin{equation} \label{uce}
\fghat_{R} := \fg_{R} \oplus \Omega^1_{R} / \d R ,
\end{equation}
where $\Omega^1_{R}$ denotes the K\"{a}hler differentials of $R/\CC$ and $\d: R \rightarrow \Omega^{1}_{R}$ is the universal derivation.
The Lie bracket is defined by
\[
[X \otimes r, Y \otimes s] =  [X,Y] \otimes rs + \overline{\langle X, Y \rangle s dr} . 
\]
The first term is an element of $\fg \tensor R$ and second terms lies in the summand $\Omega^1_{R} / \d R$. 
We have an exact sequence of $R$-modules
\[
\ses{\Omega^1_{R} / \d R}{\fghat_{R}}{\fg_{R}},
\]
exhibiting $\fghat_R$ as a central extension of $\fg \tensor R$. 

\begin{rmk}
When $\fg$ is semi-simple \brian{Killing form}.
\end{rmk}

\begin{eg}

Let $n \geq 0$ be an integer.
An important class of examples is obtained by taking
\[
R := \CC[t^{\pm 1}_0, \cdots, t^{\pm -1}_n] .
\]
This is the algebra of functions on the $(n+1)$-dimensional algebraic torus. 

When $n=0$, the vector space $\Omega^1_R / \d R$ is one-dimensional with an explicit isomorphism given by the residue
\[
{\rm Res} : \Omega^1_R / \d R \xto{\cong} \CC . 
\]
The resulting Lie algebra $\fghat_R$ is the ordinary affine algebra usually denoted by $\Hat{\fg}$. 
 
For any $n \geq 1$ the vector space $\Omega^{1}_R / \d R$ is infinite dimensional. 
Indeed, let us denote $k_i = t_i^{-1} \d t_i$. 
The space $\Omega^1_{R} / \d R$ is generated over the ring $\CC[t_0^{\pm 1}, \ldots, t_n^{\pm 1}]$ by the symbols $k_0,\ldots, k_n$ subject to the relation
\[
\sum_{i = 0}^n m_i t_0^{m_0} \cdots t_n^{m_n} k_i = 0
\]
where $(m_0,\ldots, m_n)$ is any $n$-tuple of integers.
The Lie algebra $\Hat{\fg}_R$ is called the $(n+1)$-{\em toroidal} Lie algebra associated to $\fg$.

\end{eg}

\matt{Something about $\Linf$ algebras, maybe in an appendix ?}

It will be useful for us to have an $L_\infty$-model for the Lie algebra $\fghat_{R}$. 
This model amounts to replacing the $R$-module $\Omega^1_R / \d R$ appearing as the central term by the cochain complex
\[
\cK_{R} = \Ker (\d) [2] \to R[1] \xto{\d} \Omega^{1}_{R} .
\]
Just as the Lie algebra $\fghat_R$ is a central extension of $\fg_R = \fg \tensor R$, the $L_\infty$ model we wish to construct is a central extension of $\fg_R = \fg \tensor R$ by the cochain complex $\cK_R$. 
 
The central extension is determined by a cocycle $\phi \in C^*(\fg_R, \cK_R)$ of total degree two.
The cocycle is of the form $\phi = \phi^{(0)} + \phi^{(1)}$ where
\[
\begin{array}{ccccc}
\phi^{(1)} & : & (\fg_{R})^{\otimes 2} & \rightarrow & \Omega^1_{R} \\
& & (X \otimes r) \otimes (Y \otimes s) & \mapsto & \langle X, Y \rangle (s \d r - r \d s)
\end{array}
\]
and 
\[
\begin{array}{ccccc}
\phi^{(0)} & : & (\fg_{R})^{\otimes 3} & \rightarrow & R \\
& & (X \otimes r)\otimes(Y \otimes s) \otimes (Z \otimes t) & \mapsto &\langle [X,Y], Z \rangle rst
\end{array}
\]

\begin{lemma}
The functional $\phi$ defines a cocycle in $C^*(\fg_{R}, \cK_{R}) $ of total degree two.
\end{lemma}
\begin{proof}
The differential in the cochain complex $C^*(\fg_R , \cK_R)$ is of the form $\d + \d_{CE}$ where $\d$ is the de Rham differential defining the complex $\cK_R$, and $\d_{CE}$ is the Chevalley-Eilenberg differential encoded by the Lie bracket of $\fg_R$.
Immediately, we find
\[
(\d_{CE}\phi^{(1)})(X \tensor r, Y \tensor s, Z \tensor t) = .... 
\]
Similarly, $(\d \phi^{(0)})(X \tensor r, Y \tensor s, Z \tensor t) = ...$
\brian{check signs and conventions.}
The formulas above imply $(\d_{CE} + \d) \phi = 0$ as desired.
\end{proof}

The cocycle $\phi$ defines an $L_{\infty}$ central extension
\[
\cK_R \to \Tilde{\fg}_R \to \fg_R .
\]
As an $R$-module, $\fgtil_{R} = \fg_R \oplus \cK_R$, and the $L_\infty$ operations are defined by $\ell_1 = \d, \ell_2 = [,] + \phi^{(1)}$, and $\ell_3 = \phi^{(0)}$.
The following is immediate from our definitions:

\begin{lemma}
There is an isomorphism of Lie algebras
$H^{*}(\fgtil_{\R}, \ell_1) = \fghat_{\R}.$
\end{lemma} 
\begin{proof}
The cohomology of $\Tilde{\fg}_R$ is concentrated in degree zero, and as an $R$-module is given by $\fg_R \oplus H^0(\cK_R) = \fg_R \oplus \Omega^1_R / \d R$. 
It is clear from our definition of $\phi$ that the resulting Lie bracket is the same as that of $\Hat{\fg}_R$. 
\end{proof}

\subsection{Vertex algebras}

We proceed to briefly recall the basics of vertex algebras and discuss an important class of examples, which will later be constructed geometrically via factorization algebras. We refer the reader to ~\cite{FBZ, Kac} for details. 

\begin{dfn}
A vertex algebra $(V,\vac, T, Y)$ is a complex vector
space $V$ along with the following data:
\begin{itemize}
\item A vacuum vector $\vac \in V$.
\item A linear map $T : V \to V$ (the translation operator).
\item A linear map $Y(-,z) : V \to {\rm End}(V)\llbracket z^{\pm 1}
  \rrbracket$ (the vertex operator). We write $Y(v,z) = \sum_{n \in \mathbb{Z}} A_n^v z^{-n}$
  where $A_n^v \in {\rm End}(V)$. 
\end{itemize} 
satisfying the following axioms:
\begin{itemize}
\item For all $v,v' \in V$ there exists an $N \gg 0$ such that $A_n^v
  v' = 0$ for all $n > N$. (This says that $Y(v,z)$ is a {\it field}
  for all $v$). 
\item (vacuum axiom) $Y(\vac, z) = {\rm id}_V$ and
    $Y(v,z)  |0\> \in v + z V \llbracket z \rrbracket$ for all
    $v \in V$. 
\item (translation) $[T,Y(v,z)] = \partial_z Y(v,z)$ for all $v \in
  V$. Moreover $T \vac = 0$. 
\item (locality) For all $v,v' \in V$, there exists $N \gg 0$ such
  that 
\[
(z-w)^N[Y(v,z),Y(v',w)] = 0
\]
in ${\rm End}(V) \llbracket z^{\pm 1},w^{\pm 1}\rrbracket$. 
\end{itemize}
\end{dfn}


In order to prove that a given $(V,\vac,T,Y)$ forms a vertex algebra, the following "reconstruction" or "extension" theorem is very useful. It shows that any collection of local fields generates a vertex algebra in a suitable sense. 

\matt{we should throw in some references to Li and friends, since this theorem has been "discovered" in a number of different papers}

\begin{theorem}[~\cite{FBZ}, ~\cite{DSK}] \label{rec_thm} Let $V$ be a complex vector space equipped with: an
  element $\vac \in V$, a linear map $T : V \to V$, a 
    set of vectors $\{a^s\}_{s \in S} \subset V$ indexed by a set $S$, and fields $A^s(z) =
    \sum_{n \in \mathbb{Z}} A_n^sz^{-n-1}$ for each $s\in S$ such that:
\begin{itemize}
\item For all $s \in S$, $A^s(z) \vac \in a^s + z V\llbracket
    z\rrbracket$;
\item $T \vac = 0$ and $[T,A^s(z)] = \partial_z A^s(z)$;
\item $A^s(z)$ are mutually local;
\item and $V$ is spanned by $\{A_{j_1}^{s_1} \cdots A_{j_m}^{s_m}
  |0\>\}$ as the $j_i's$ range over negative integers. 
\end{itemize}
Then, the data $(V,\vac, T,Y)$ defines a unique vertex algebra satisfying 
\[
Y(a^s,z) = A^s(z) .
\]
\end{theorem}

\begin{rmk}
The version stated above appears in ~\cite{DSK}, and is slightly more general than the version stated in ~\cite{FBZ}. 
\end{rmk}

\subsection{The vertex algebras $V_{k}(\fg)$ and $V(\fghat_{\R})$}

Many vertex algebras are constructed from vacuum representations of affine Lie algebras and their generalizations. We proceed to review the vertex algebra structure on the affine Kac-Moody vacuum module $V_{k}(\fg)$ and extend the construction to vacuum representations of $\fghat_{R}$, where $R=A[t,t^{-1}]$ for some $\CC$-algebra $A$.

\subsubsection{$V_{k}(\fg)$}
Let $\fghat = \fg[t,t^{-1}] \oplus \CC k$ be the affine Kac-Moody algebra, $\fghat^{+} = \fg[t] \oplus \CC k$ denote the positive sub-algebra, and $\CC[k]$ denote the representation of $\fghat^{+}$ on which $\fg[t]$ acts by $0$ and $k$ acts by multiplication. For $J \in \fg$, denote $J \otimes t^n$ by $J_n$, and $1 \in \CC[k]$ by $\vac$

It is well-known (see for instance ~\cite{FBZ}) that the induced vacuum representation
\[
V_{\lambda}(\fg) := \on{Ind}^{\fghat}_{\fghat^+} \CC[k]
\] 
has a $\CC[k]$-linear vertex algebra structure, which is generated, in the sense of the above reconstruction theorem, by the fields
\[
J^i (z) := Y(J^{i}_{-1} \vac, z) = \sum_{n \in \mathbb{Z}} J^i_n z^{-n-1},
\]
where $\{ J^{i} \}$ is a basis for $\fg$. These satisfy the commutation relations
\[
[J^i(z), J^{k}(w)] = [J^{i}, J^{k}](w) \delta(z-w) +  \langle J^{i}, J^{k} \rangle k \partial_w \delta(z-w)
\]
where 
\[
\delta(z-w) = \sum_{m} z^{m} w^{-m-1} 
\]
The translation operator $T$ is determined by the properties
\[
T \vac =0, [T, J^i_n] = -n J^{i}_{n-1}.
\]


\subsubsection{$V(\fghat_{\R})$}

The vertex algebra structure on $V_{k}(\fg)$ may be generalized to vacuum representations of $\fghat_{\R}$, in the case where $\R = A[t,t^{-1}] := A \otimes \CC[t,t^{-1}]$ for some $\CC$-algebra $A$. 


Let $A$ be a $\CC$-algebra, $\R=A[t,t^{-1}] := A \otimes \CC[t,t^{-1}]$, and $\fghat_{\R}$ the Lie algebra (\ref{uce}). 
%We will assume that $A$ has a countable $\mathbb{C}$-basis $\{ f^r \}_{r \in \mathbb{N}}$, and so $\{ f^r t^s \}_{r \in \mathbb{N}, s \in \mathbb{Z}} $ is a countable basis for $A[t,t^{-1}]$. 
% If $J \in \fg$, we denote by $J_n f^r$ the element $J \otimes f^r t^{n} \in \fg \otimes A[t,t^{-1}]$. 
There is a natural injection
\[
\Omega^1_{A[t]}/ d A[t] \hookrightarrow \Omega^{1}_{\R} / d \R
\]
and 
$$\fghat^{+}_{\R} := \fg[t] \oplus \Omega^1_{A[t]}/ d A[t]  $$ is naturally a Lie subalgebra of $\fghat_{\R}$. Let $\mathbf{C}$ denote the trivial representation of $\fghat^{+}_{\R}$, and let
\begin{equation}
V(\fghat_{\R}) := \on{Ind}^{\fghat_{\R}}_{\fghat^+_{\R}} \mathbf{C}
\end{equation}
Let $\vac := 1 \in \mathbf{C}$, and consider the field assignments
\begin{align}
J_f(z) := Y(J^{i} \otimes f t^{-1} \vac,z ) & := \sum_{n \in \mathbb{Z}} (J^i  \otimes f t^{n}) z^{-n-1}, \\
K_{f \frac{dt}{t}} :=  Y(t^{-1} f dt \vac,z ) & := \sum_{n \in \mathbb{Z}} ( f t^{n-1} dt )  z^{-n}, \\
K_{t^{-1} \omega} := Y(t^{-1}  \omega \vac,z ) & := \sum_{n \in \mathbb{Z}}( t^{n} \omega )    z^{-n-1} \\
\end{align}
where $J\in \fg, f \in A, \omega \in \Omega^{1}_A$. 

The commutation relations between these fields are easily checked to be
$$
[J^1_f (z), J^2_g (w)] = \left( [J^1, J^2]_{fg} (w) + \langle J^1, J^2 \rangle  K_{t^{-1} g df} (w)      \right) \delta(z-w) + \langle J^1, J^2 \rangle K_{fg \frac{dt}{t}} (w) \partial_w \delta(z-w)
$$
$$
[J(z), K_{f \frac{dt}{t}}(w) ] = [J(z), K_{t^{-1} \omega} (w)] = [ K_{f \frac{dt}{t}}(z),  K_{t^{-1} \omega} (w)] =0
$$

The operator $T$, corresponding to the Lie derivative $L_{-\partial_t}$, is defined by
\[
T \vac =0, \; [T, J^i \otimes f t^n] = -n J^i \otimes f t^{n-1}, \; [T, f t^n dt] = -n f t^{n-1} dt, \; [T, t^n \omega] = -n t^{n-1} \omega
\]

\begin{theorem}
The above field assignments, together with $T$ equip $V(\fghat_{\R})$ with the structure of a vertex algebra.
\end{theorem}

\begin{proof}
We begin by checking that the field assignment above is well-defined. This amounts to verifying that $Y(d(t^{-1}f)\vac,z ) =0$.  We have 
\begin{align*}
Y(d(t^{-1}f) \vac, z) &= Y(t^{-1} df \vac, z) - Y(f t^{-2} f \vac, z) \\
&= Y(t{-1} df \vac, z) - Y([T, t^{-1}f dt  ] \vac, z) \\
&= Y(t^{-1} df \vac, z) - \partial_z Y( t^{-1} f dt \vac, z) \\
&= \sum_{n} (t^n df + n t^{n-1} f) z^{-n-1} = \sum_{n} d(t^n f) z^{-n-1} = 0
\end{align*}
The result now follows by applying the reconstruction theorem \ref{rec_thm} to $V(\fghat_\R)$ and the fields $\{ J_f (z), K_{f \frac{dt}{t}}, K_{t^{-1} \omega} \}$ for $f \in A, \omega \in \Omega^{1}_A $.
\end{proof}

\section{(Pre)-factorization algebras and examples}

In this section we briefly recall basic notions pertaining to pre-factorization algebras and their relationship with vertex algebras. We refer the reader to \cite{CG} for details. Our summary closely follows the presentation given in \cite{} {\color{red} Owen-Kasia}. 

Let $M$ be a smooth manifold, and $\symcat$ a symmetric monoidal category. 
%Let $\Open(M)$ denote the poset category whose objects are opens in $M$ and where a morphism is an inclusion.
%A factorization algebra will be a functor from $\Open(M)$ to a symmetric monoidal category $\bC ^\otimes$ equipped with further data and satisfying further conditions.
%We will explain this extra information in stages.
%(Note that almost all the definitions below apply to an arbitrary topological space, or even site with an initial object, and not just smooth manifolds.)

\begin{dfn}
A {\it prefactorization algebra} $\Fcal$ on $M$ with values in $\symcat$ consists of the following data:
\begin{itemize}
\item for each open $U \subset M$, an object $\Fcal(U) \in \symcat $,
\item for each finite collection of pairwise disjoint opens $U_1,\ldots,U_n$ and an open $V$ containing every $U_i$, a morphism
\[
\Fcal(\{U_i\}; V): \Fcal(U_1) \otimes \cdots \otimes \Fcal(U_n) \to \Fcal(V),
\]
\end{itemize}
and satisfying the following conditions:
\begin{itemize}
\item composition is associative, so that the triangle
\[
\begin{tikzcd}
\bigotimes_i \bigotimes_j \Fcal(T_{ij}) \arrow{rr} \arrow{rd} &&\bigotimes_i \Fcal(U_{i}) \arrow{ld} \\
&\Fcal(V)&
\end{tikzcd}
\]
commutes for any collection $\{U_i\}$, as above, contained in $V$ and for any collections $\{T_{ij}\}_j$ where for each $i$, the opens $\{T_{ij}\}_j$ are pairwise disjoint and each contained in $U_i$,
\item the morphisms $\Fcal(\{U_i\}; V)$ are equivariant under permutation of labels, so that the triangle
\[
\begin{tikzcd}
\Fcal(U_{1}) \otimes \cdots \otimes \Fcal(U_n) \arrow{rr}{\simeq} \arrow{rd} && \Fcal(U_{\sigma(1)}) \otimes \cdots \otimes \Fcal(U_{\sigma(n)}) \arrow{ld}\\
&\Fcal(V)&
\end{tikzcd}
\]
commutes for any $\sigma \in S_n$.
\end{itemize}
\end{dfn}

In this paper, we will take the target category $\symcat$ to be either $\Vect$ or $\dgVect$. 

A {\it factorization algebra} is
a prefactorization algebra satisfying a descent axiom. Descent for
ordinary sheaves (or cosheaves) says that one can recover the value of
the sheaf on large open sets by breaking it up into smaller
opens. That is, if $\scr{U} = \{U_i\}$ is a cover of $U \subset M$
then a presheaf $\Fcal$ of vector spaces is a sheaf iff 
\[
\begin{tikzcd}
\Fcal(U) \arrow[r, shift left] & \bigoplus_i \Fcal(U_i)
\arrow[r, shift left] \arrow[r, shift right] & \bigoplus_{i,j} \Fcal(U_i \cap U_j) 
\end{tikzcd}
\]
is an equalizer diagram for all opens $U$ and covers $\scr{U}$. It is convenient to introduce the \v{C}ech
complex associated to $\scr{U}$. The $p$th space is
\[
\check{C}^p(\scr{U},\Fcal) := \bigoplus_{i_0,\ldots,i_p}
\Fcal(U_{i_0} \cap \cdots \cap U_{i_p}) .
\]
The differential $\check{C}^p \to \check{C}^{p+1}$ is induced from the
natural inclusion maps $$U_{i_0} \cap \cdots \cap U_{i_p}
\hookrightarrow U_{i_0} \cap \cdots \cap \Hat{U}_{i_j} \cap \cdots
\cap U_{i_p}. $$ The sheaf condition is equivalent to saying that the
natural map
\[
\begin{tikzcd}
\Fcal(U) \to {\rm H}^0(\check{C}(\scr{U},\Fcal)) 
\end{tikzcd}
\]
is an isomorphism. There is a similar construction for cosheaves, but
the arrow goes in the opposite direction. 

The descent condition for factorization algebras is formulated with respect to a different topology, that is, for
only a special class of open covers. Call an open cover $\scr{U} = \{U_i\}$ of
$U \subset M$ a {\it Weiss cover} if for any finite collection of points
$\{x_1,\ldots,x_k\}$ in $U$, there exists an open set $U_i$ such that
$\{x_1,\ldots,x_k\} \subset U_i$. This is equivalent to providing a
topology on the Ran space. 

A Weiss cover defines a Grothendieck topology on ${\rm
    Op}(M)$, the poset of opens in $M$. 
    
\begin{dfn}    
    A {\it factorization algebra}
  on $M$
  is a prefactorization algebra on $M$ that is, in addition, a
  homotopy cosheaf for the Weiss topology. 
\end{dfn}  

When $\scr{C}^\tensor = {\rm dgVect}$ we can be explicit about this
homotopy gluing condition using a variant of the \v{C}ech complex
above. Let $\Fcal$ be a cosheaf of dg vector spaces. For $\scr{U} = \{U_i\}_{i \in I}$ let
$\check{C}^p(\scr{U}, \Fcal)$ be the complex
\[
\begin{tikzcd}
\bigoplus_{i_0,\ldots,i_p} \Fcal(U_{i_1} \cap \cdots \cap U_{i_k})
[p-1] 
\end{tikzcd}
\]
with differential inherited from $\Fcal$. Then
$\check{C}(\scr{U},\Fcal)$ is a bigraded object. The differential is the total differential obtained from
combining the ordinary
\v{C}ech differentials above plus the internal differential of
$\Fcal$. The cosheaf
condition is that the natural map
\[
\begin{tikzcd}
\check{C}(\scr{U},\Fcal) \to \Fcal(U)
\end{tikzcd}
\]
is an equivalence for all Weiss covers $\scr{U}$ of $U$. 

\subsection{Factorization Lie algebras and factorization envelopes}

An important class of examples of (pre)factorization algebras is given by the \emph{factorization envelope} construction, which we proceed to review in this section.

Let $\cL$ be a fine sheaf of Lie algebras on $M$. 

\matt{explain factorization envelopes}

\begin{enumerate}
\item Factorization Lie algebras
\item Compactly supported sections of a local Lie algebra is a factorization Lie algebra
\item Pushforward of local Lie algebra is a factorization Lie algebra
\item factorization envelopes as examples of factorization algebras
\item relations with vertex algebras in one dimension. 
\end{enumerate}



\subsection{Translation-invariant (pre)factorization algebras}

Suppose now that $\sF$ is prefactorization algebra on $\mathbb{R}^n$ n the appropriate category of
differentiable vector spaces. \footnote{Some care is needed to
  define this category correctly. We refer the interested reader to
  \cite{CG}}. $\mathbb{R}^n$ acts on itself by translations. For an open subset $U \subset \mathbb{R}^n$ and $x \in \mathbb{R}^n$, let 
\[
\tau_x U := \{ y \in \mathbb{R}^n \vert y-x \in U \}
\]
Clearly, $\tau_x (\tau_y U) = \tau_{x+y} U$. We say that $\sF$ is \emph{translation-invariant} if we are given isomorphisms
\[
\phi_x : \sF(U) \rightarrow \sF(\tau_x U)
\]
for each $x \in \mathbb{R}^n$ compatible with composition and the structure maps of $\sF$. We refer to section 4.8 of  \cite{CG} for details. 

\begin{eg}
For any Lie algebra $\fg$, the factorization envelope $\U(\fg \otimes \Omega^*_{\mathbb{R}^n})$ of section ?? is translation-invariant. 
\end{eg}

If $\sF$ is a
prefactorization algebra on $\CC^n$, we can further refine the notion of translation-invariance. 
We say that $\sF$ is {\it holomorphically translation invariant}
if
\begin{itemize}
\item $\sF$ is translation invariant. 
\item There exists a degree $-1$ derivation $\eta: \sF \to
  \sF$ such that $\dd \eta = \partial_{\Bar{z}}$ as derivations of
  $\sF$. 
\end{itemize}

This condition means that anti-holomorphic vector fields act homotopically trivially on $\sF$. 

\begin{eg}
For any Lie algebra $\fg$, the holomorphic factorization envelope $\U(\fg \otimes \Omega^{0,*}_{\CC^n})$ is holomorphically translation-invariant, with \matt{elaborate what $\eta$ is here}.
\end{eg}


\subsection{$S^1$-equivariant factorization algebras on $\CC$ and vertex algebras}

In order to build a bridge between the world of prefactorization algebras on $\CC$ and vertex algebras, we will need to consider prefactorization algebras possessing a type of $S^1$-equivariance, which we proceed to review. 

The last piece of data we need corresponds to the ``conformal
decomposition'' of a vetex algebra. For us, this will come from an
$S^1$-action on $\sF$. The reader is encouraged to look at
\cite{CG} for a precise definition, but we assume that we have a {\it
  nice} action of $S^1$ on $\sF$ and it is compatible with the
translation invariance discussed above. 

We can now read off the data of the
vertex algebra from $\sF$:

\begin{itemize}
\item Let $\sF^{(l)}(r) \subset \sF(r)$ be the $l$th eigenspace for the $S^1$-action. The underlying vector
space for the vertex algebra is
\[
V := \bigoplus_l {\rm H}^*(\sF^{(l)}(r)) .
\]
\item The translation operator. The action of $\partial_z$ on $\sF^{(l)}(r)$ has the form
\[
\partial_z : \sF^{(l)}(r) \to 
\sF^{(l-1)}(r) .
\]
We let $T : V \to V$ be the operator which is $\partial_z$ restricted
to the $l$-th eigenspace. 
\item The fields. Consider the map
\[
\mu_{z,0} : \left(\lim_{r \to 0} {\rm
    H}^*(\sF(r))\right)^{\tensor 2} \to {\rm Hol}\left({\rm
  Conf}_2(\CC), {\rm H}^*(\sF(\CC))\right)
\]
defined above. Certainly, we have a map $V
  \to \lim_{r \to 0} {\rm
    H}^*(\sF(r))$, so it makes sense to resctict $\mu_{z,0}$ to a
  map 
\[
V \tensor V \to {\rm Hol}\left({\rm Conf}_2(\CC), {\rm
    H}^*(\sF(\CC))\right) \simeq {\rm Hol}\left(\CC^\times, {\rm
    H}^*(\sF(\CC))\right) .
\]
Post composing this with the projection maps $H^*(\sF(\infty)) \to
V_l$ combine to define the map
\[
\Bar{\mu}_{z,0} : V \tensor V \to \prod_l {\rm Hol}(\CC^\times, V_l)
\]
We can perform Laurent expansions to view this as
\[
\Bar{\mu}_{z,0} : V \tensor V \to \Bar{V} \llbracket z^{\pm 1}
\rrbracket .
\]
We define $Y(-,z) : V \to {\rm End}(V)\llbracket z^{\pm 1}\rrbracket$ by
\[
Y(v,z) v' \; := \; \Bar{\mu}_{z,0}(v,v') .
\]
One can show that this actually lies in $V((z))$ for all $v,v'$. 
\end{itemize}

The above can be made much more precise and made into the following
theorem. 

\begin{theorem}[Theorem 5.2.2.1 \cite{CG1}] \label{fv} Let $\sF$ be a $S^1$-equivariant holomorphically translation invariant factorization algebra on $\CC$. Suppose
\begin{itemize}
\item The action of $S^1$ on $\sF(r)$ extends smoothly to an action of the algebra of distributions on $S^1$. 
\item For $r < r'$ the map 
\[
\sF^{(l)}(r) \to \sF^{(l)}(r')
\]
is a quasi-isomorphism.
\item The cohomology ${\rm H}^*(\sF^{(l)}(r))$ vanishes for $l \gg 0$.
\item For each $l$ and $r > 0$ we require that ${\rm H}^*(\sF^{(l)}(r))$ is isomorphic to a countable sequential colimit of finite dimensional vector spaces. 
\end{itemize}
Then $\mathbb{V}{\rm ert} (\sF) := \oplus_l {\rm H}^*(\sF^{(l)}(r))$ (which is independent of $r$ by assumption) has the structure of a vertex algebra.
\end{theorem}

Let ${\rm PreFact}_\CC$ denote the category of prefactorization
algebras on $\CC$. Let ${\rm PreFact}^{\rm hol}_\CC \subset {\rm
  PreFact}_\CC$ be the full subcategory spanned by prefactorization
algebras satisfying the conditions of the above theorem. This result
can be upgraded to provide a functor
\[
\mathbb{V}{\rm ert} : {\rm PreFact}^{\rm hol}_\CC \to {\rm Vert}
\]
where ${\rm Vert}$ is the category of vertex algebras. 

\section{Factorization algebras from holomorphic fibrations}

In this section, we describe our main construction factorization algebras from locally trivial holomorphic fibrations. 

\subsection{The local case - trivial fibrations}

Let us begin with the following data:
\begin{itemize}
\item a complex manifold $F$
\item $X = \CC^n, n \geq 1$
\item $E = X \times F$, with $\pi: E \mapsto X $ the canonical projection
\item $(\fg, \langle, \rangle)$ a Lie algebra with an invariant bilinear form.
\end{itemize}

We note that $\fg_E := (\fg \otimes \Omega^{0,*}_E, \delbar)$ forms a sheaf of DGLA's on $E$, and
set
\[
\cK_E := \on{Tot} ( \Omega^{0,*} \overset{\partial}{\longrightarrow} \Omega^{1,*})
\]
(i.e. $\cK_E$ is the totalization of the double complex of forms $\Omega^{p,q}_E$ with $p \leq 1$), Denote by $d_{\cK_E}$ the total differential of $\cK_E$. 
Let 

$$ \phi^{(1)}: (\fg_{E})^{\otimes 2} \rightarrow \Omega^(1,*)_{E} $$
$$ \phi^{(1)} ((X \otimes r) \otimes (Y \otimes s)) = \langle X, Y \rangle (s \del r - r \del s) $$
and 
$$ \phi^{(0)}: (\fg_{E})^{\otimes 3} \rightarrow \Omega^{0,*}_E $$
$$ \phi^{(0)}( (X \otimes r)\otimes(Y \otimes s) \otimes (Z \otimes t)) = \langle [X,Y], Z \rangle rst $$
We may view $\phi = \phi^{(0)} + \phi^{(1)}$ as an cochain in the cohomological Chevalley-Eilenberg complex $$ \CEcoh(\fg_{E}, \cK_{E}) $$ of total degree $2$.

\begin{lemma}
$\phi$ defines a cocycle in $\CEcoh(\fg_{\E}, \cK_{E}) $ of total degree $2$.
\end{lemma}
\begin{proof}

\end{proof}

We may use the cocycle $\phi$ to equip 
\begin{equation}
\fghat_{E} := \fg \otimes \Omega^{0,*}_E \oplus \cK_{E}
\end{equation}
with the structure of a local $L_{\infty}$ central extension. Explicitly, this means that $\wt{d} = d + d_{CE} + \phi$ is a differential of cohomological degree $1$ on the complex of sheaves on $E$
\[
C^{*}(\fghat_E) := \on{Sym}(\fghat_E[1]),
\]
where 
\begin{itemize}
\item $d = \delbar + d_{\cK_E}$ is the differential on $\fghat_E$ (with the summands acting on $\fg_E, \cK_E$ respectively)
\item $d_{CE}$ is the Chevalley-Eilenberg differential of $\fg_E$
\item $\phi$ is extended to $\on{Sym}(\fghat_E[1])$ as a derivation
\end{itemize}

Let 
\begin{equation}
\fghat_{X,E} = \pi_*(\fghat_E)
\end{equation}


\begin{lemma}
$\fghat_{X,E}$ has the structure of a pro-local $L_{\infty}$--algebra on $X$
\end{lemma}

\subsubsection{The factorization algebra $\sF_{E,X}$}

We proceed to construct a factorization algebra on $X$ from $\fghat_{X,E}$. For an open set $U \subset X$, let
\begin{equation}
\fghat^c_{X,E}(U) = \Gamma_c (U, \fghat_{X,E})
\end{equation}
be the co-sheaf of sections with compact support of $\fghat_{X,E}$. $\fghat^c_{X,E}$ has the structure of a complex of pre-cosheaves of $L_{\infty}$ algebras on $X$. 

Let
\begin{equation}
\sF_{X,E} := C_{*} ( \fghat_{X,E} )
\end{equation}
be the corresponding homology Chevalley-Eilenberg compex, which for each open $U \subset X$ assigns the complex
\[
\sF_{X,E}(U) = C_{*} (\fghat^c_{X,E}(U))
\]

\begin{theorem}
$\sF_{X,E}$ has the structure of a holomorphically translation-invariant factorization algebra on $X=\CC^n$. 
\end{theorem}

\subsection{$dim(X)=1$ and vertex algebras}

\section{Factorization homology}

{\color{red}
\begin{enumerate}
\item Discuss factorization algebra with base a compact complex manifold and potentially non-trivial torus bundle
\item What is factorization homology or equivalently conformal blocks
\item Compute factorization homology for compact base and tirivial/non-trivial torus bundle
\end{enumerate}
}

\newpage

\bibliographystyle{alpha}
%\bibliographystyle{spmpsci}  
\bibliography{toroidal_bib}



\address{\tiny DEPARTMENT OF MATHEMATICS AND STATISTICS, BOSTON UNIVERSITY, 111 CUMMINGTON MALL, BOSTON} \\
\indent \footnotesize{\email{szczesny@math.bu.edu}}

\address{\tiny DEPARTMENT OF MATHEMATICS, NORTH\brian{WESTERN/EASTERN},...}
\indent \footnotesize{\email{brianwilliams.math@gmail.com}}

\end{document}
