\documentclass[12pt]{amsart}

\usepackage{slashed}
\linespread{1.25}

\usepackage{amsrefs, mathrsfs,amsmath, amscd, amsthm,amssymb, amsfonts, verbatim,subfigure, enumerate}
%\usepackage[mathcal]{eucal}
\usepackage[mathscr]{euscript}
\usepackage[all,cmtip]{xy}
\usepackage{graphicx}
\usepackage[top=1in, bottom=1.25in, left=1.3in, right=1.3in]{geometry}
%\usepackage{fullpage}
%\usepackage{epstopdf}
%\DeclareGraphicsRule{.tif}{png}{.png}{`convert #1 `basename #1 .tif`.png}
%Use Palatino font

\usepackage{mathpazo}
\usepackage[colorlinks=true,linkcolor=blue,citecolor=red]{hyperref}

\usepackage{tikz}
\usetikzlibrary{arrows,shapes}
\usetikzlibrary{trees}
\usetikzlibrary{matrix,arrows}
\usetikzlibrary{positioning}
\usetikzlibrary{calc,through}
\usetikzlibrary{decorations.pathreplacing}
\usepackage{pgffor}
%\usepackage{tikz-feynman} 

\usetikzlibrary{decorations.pathmorphing}
\usetikzlibrary{decorations.markings}
\tikzset{
	% >=stealth', %%  Uncomment for more conventional arrows
    vector/.style={decorate, decoration={snake}, draw},
	provector/.style={decorate, decoration={snake,amplitude=2.5pt}, draw},
	antivector/.style={decorate, decoration={snake,amplitude=-2.5pt}, draw},
    fermion/.style={draw=black, postaction={decorate},
        decoration={markings,mark=at position .55 with {\arrow[draw=black]{>}}}},
    fermionbar/.style={draw=black, postaction={decorate},
        decoration={markings,mark=at position .55 with {\arrow[draw=black]{<}}}},
    fermionnoarrow/.style={draw=black},
    gluon/.style={decorate, draw=black,
        decoration={coil,amplitude=4pt, segment length=5pt}},
    scalar/.style={dashed,draw=black, postaction={decorate},
        decoration={markings,mark=at position .55 with {\arrow[draw=black]{>}}}},
    scalarbar/.style={dashed,draw=black, postaction={decorate},
        dwecoration={markings,mark=at position .55 with {\arrow[draw=black]{<}}}},
    scalarnoarrow/.style={dashed,draw=black},
    electron/.style={draw=black, postaction={decorate},
        decoration={markings,mark=at position .55 with {\arrow[draw=black]{>}}}},
	bigvector/.style={decorate, decoration={snake,amplitude=4pt}, draw},
}

\pagestyle{plain}
\newtheorem{thm}{Theorem}[section]
\newtheorem{prop}[thm]{Proposition}
\newtheorem{lemma}[thm]{Lemma}
\newtheorem{cor}[thm]{Corollary}
\newtheorem{claim}[thm]{Claim}

\theoremstyle{definition}
\newtheorem{dfn}[thm]{Definition}
\newtheorem{dfn/lem}{Definition/Lemma}
\newtheorem{lesson}[thm]{Lesson}

\theoremstyle{remark}
\newtheorem{rmk}[thm]{Remark}
\newtheorem{eg}[thm]{Example}
\newtheorem{construction}[thm]{Construction}


\linespread{1.25}

\usepackage{parskip}
\setlength{\parindent}{18pt}
\setlength{\parindent}{0cm}


\usepackage{color}   %May be necessary if you want to color links
\usepackage{hyperref}
\hypersetup{
    colorlinks=true, %set true if you want colored links
    linktoc=all,     %set to all if you want both sections and subsections linked
    linkcolor=blue,  %choose some color if you want links to stand out
}


%\numberwithin{equation}{section}
%\numberwithin{example}{section}
%\numberwithin{definition}{section}


%%%%%%%%%%%%%%%%%%%%%%         Defintions        %%%%%%%%%%%%%%%%%%%%%%%%%%%%%%%%%%
\newcommand{\on}{\operatorname}
%\newcommand{\C}{\mathbb{C}}
\newcommand{\N}{\mathbb{N}}
\newcommand{\R}{\mathcal{R}}
\newcommand{\Q}{\mathbb{Q}}
\newcommand{\Z}{\mathbb{Z}}
\newcommand{\Etau}{{\text{E}_\tau}}
\newcommand{\E}{{\mathcal E}}
%\newcommand{\F}{\mathbf{F}}
%\newcommand{\G}{\mathbf{G}}
\newcommand{\eps}{\epsilon}
%\newcommand{\g}{\mathbf{g}}
\newcommand{\im}{\op{im}}
%%%%%%%%%%%%%%%%%%%%%%         Functions         %%%%%%%%%%%%%%%%%%%%%%%%%%%%%%%%%%%
\providecommand{\abs}[1]{\left\lvert#1\right\rvert}
\providecommand{\norm}[1]{\left\lVert#1\right\rVert}
\newcommand{\abracket}[1]{\left\langle#1\right\rangle}
\newcommand{\bbracket}[1]{\left[#1\right]}
\newcommand{\fbracket}[1]{\left\{#1\right\}}
\newcommand{\bracket}[1]{\left(#1\right)}
\providecommand{\from}{\leftarrow}
\newcommand{\bl}{\textbf}
\newcommand{\mbf}{\mathbf}
\newcommand{\mbb}{\mathbb}
\newcommand{\mf}{\mathfrak}
\newcommand{\mc}{\mathcal}
\newcommand{\cinfty}{C^{\infty}}
\newcommand{\pa}{\partial}
\newcommand{\prm}{\prime}
%\renewcommand{\dbar}{\bar\pa}
\newcommand{\OO}{{\mathcal O}}
\newcommand{\hotimes}{\hat\otimes}
\newcommand{\BV}{Batalin-Vilkovisky }
\newcommand{\CE}{Chevalley-Eilenberg }
\newcommand{\suml}{\sum\limits}
\newcommand{\prodl}{\prod\limits}
\newcommand{\into}{\hookrightarrow}
\newcommand{\Ol}{\mathcal O_{loc}}
\newcommand{\mD}{{\mathcal D}}
\newcommand{\iso}{\cong}
\newcommand{\dpa}[1]{{\pa\over \pa #1}}
\newcommand{\PP}{\mathrm{P}}
\newcommand{\Kahler}{K\"{a}hler }
\newcommand{\Fock}{{\mathcal Fock}}



\newcommand{\ol}{\overline}
\newcommand{\nc}{\newcommand}
\nc{\wt}{\widetilde}
\nc{\CEcoh}{\mathcal{C}^{*}}



\nc{\h}{\mathfrak{h}}
\nc{\g}{\mathfrak{g}}
\nc{\ghat}{\widehat{\g}}
\nc{\n}{\mathfrak{n}}
\nc{\F}{\mc{F}}
\nc{\C}{\mathbb{C}}
\nc{\delbar}{\overline{\partial}}
\nc{\del}{\partial}

\nc{\gt}[1]{\g_{\tau,#1}}
\nc{\Gt}[1]{\mathfrak{G}_{\tau,#1}}
\nc{\K}{\mc{K}}
\nc{\opqm}[2]{\Omega^{#1,#2}_{m}}
\nc{\gtil}{\wt{\g}}
\nc{\CC}{\mathcal{C}_{*}}
\nc{\Sym}{\on{Sym}}
\nc{\dzbar}{d \overline{z}}

\nc{\ip}{\langle \bullet , \bullet \rangle}
\nc{\ses}[3]{0 \rightarrow #1 \rightarrow #2 \rightarrow #3 \rightarrow 0}


\renewcommand{\Im}{\op{Im}}
\renewcommand{\Re}{\op{Re}}
%%%%%%%%%%%%%%%%%%%%%%     Math    Operators         %%%%%%%%%%%%%%%%%%%%%%%%%%%%%%%
\DeclareMathOperator{\mHom}{\mathcal{H}om}
\DeclareMathOperator{\End}{End}
\DeclareMathOperator{\Supp}{Supp}
%\DeclareMathOperator{\Sym}{Sym}
\DeclareMathOperator{\Hom}{Hom}
\DeclareMathOperator{\Spec}{Spec}
\DeclareMathOperator{\Deg}{Deg}
\DeclareMathOperator{\Diff}{Diff}
\DeclareMathOperator{\Ber}{Ber}
\DeclareMathOperator{\Vol}{Vol}
\DeclareMathOperator{\Tr}{Tr}
\DeclareMathOperator{\Or}{Or}
\DeclareMathOperator{\Ker}{Ker}
\DeclareMathOperator{\Mat}{Mat}
\DeclareMathOperator{\Ob}{Ob}
\DeclareMathOperator{\Isom}{Isom}
\DeclareMathOperator{\PV}{PV}
\DeclareMathOperator{\Der}{Der}
\DeclareMathOperator{\HW}{HW}
\DeclareMathOperator{\Eu}{Eu}
\DeclareMathOperator{\HH}{H}
\DeclareMathOperator{\Jac}{Jac}
\DeclareMathOperator{\Res}{Res}
\DeclareMathOperator{\d}{{\rm d}}






%%%%%%%%%%%%%%%%%%%%%%%%%%%%%% Allow display breaks within equations %%%%%%%%%%%%%%%%%%%%
\allowdisplaybreaks[4]  %%%%%%%%%%% choose 1-4, where 4 is the strongest desire to breack

\begin{document}

 \title{Factorization and vertex algebras from  holomorphic fibrations}
  \author{ Matt Szczesny, Jackson Walters, and Brian Williams}
  \date{}

  
  \maketitle

\begin{abstract}
Let $X$ be a complex manifold, $\pi: Y \rightarrow X$ a locally trivial holomorphic fibration with fiber $F$, and $(\g, \ip )$ a Lie algebra with an invariant symmetric form. We show how to associate to this data a holomorphic factorization algebra  $\F_{Y/X}(\g)$ on $X$ in the formalism of Costello-Gwilliam via a type of pushforward operation. When $X=\mathbb{C}$, this construction produces a vertex algebra which is a vacuum module for the universal central extension of $\g \otimes H^{0}(F, \mc{O})[z,z^{-1}]$. In particular,  when $F$ is a torus $(\C^{*})^n$, we recover a vertex algebra naturally associated to an $n+1$--toroidal algebra. We give an explicit description of the chiral homology of $\F_{Y/X}(\g)$.
\end{abstract}


%\tableofcontents


\section{Introduction}

\section{Preliminaries}
\subsection{Lie algebras and universal central extensions}

Given a complex Lie algebra $\g$ with invariant bilinear form $\ip$, and a $\C$-algebra $\R$, $ \g_{\R} := \g \otimes \R$ carries a natural Lie algebra structure with bracket
\[
[X \otimes r, Y \otimes s] = [X,Y] \otimes rs.
\]
The universal central extension $\ghat_{\R}$ of $\g_{\R}$ fits into a short exact sequence
\[
\ses{\Omega^1_{\R} / \d \R}{\ghat_{\R}}{\g_{\R}}
\]
where $\Omega^1_{\R}$ denotes the K\"{a}hler differentials of $\R/\C$ and $\d: \R \rightarrow \Omega^{1}_{\R}$ is the universal derivation. The bracket on $\ghat_{\R}$ is given by 
\[
[X \otimes r, Y \otimes s] =  [X,Y] \otimes rs + \overline{\langle X, Y \rangle r ds}
\]
where the second term lands in the quotient $\Omega^1_{\R} / d \R$. 

\begin{eg}

An important class of examples is obtained by taking
\[
\R= \R_n := \C[w^{\pm 1}_0, \cdots, w^{\pm -1}_n]
\]
to be the algebra of functions on the complex $n+1$--dimensional torus $(\C^{\times})^{n+1}$. $\ghat_{\R_n}$ is called the $n+1$--toroidal Lie algebra for $n >0$ and an affine Kac-Moody algebra for $n=0$. 

\end{eg}

We may resolve $\ghat_{\R}$ by an $L_{\infty}$ algebra $\gtil_{\R}$ as follows. Let 
\[
\K_{\R} = \R[1] \overset{d}{\rightarrow} \Omega^{1}_{\R}
\]
and let
$$ \phi^{(1)}: (\g_{\R})^{\otimes 2} \rightarrow \Omega^1_{\R} $$
$$ \phi^{(1)} ((X \otimes r) \otimes (Y \otimes s)) = \langle X, Y \rangle (r ds - s dr) $$
and 
$$ \phi^{(0)}: (\g_{\R})^{\otimes 3} \rightarrow R[1] $$
$$ \phi^{(0)}( (X \otimes r)\otimes(Y \otimes s) \otimes (Z \otimes t)) = \langle [X,Y], Z \rangle rst $$
We may view $\phi = \phi^{(0)} + \phi^{(1)}$ as an cochain in the cohomological Chevalley-Eilenberg complex $$ \CEcoh(\g_{\R}, \K_{\R}) $$ of total degree $2$.

\begin{lemma}
$\phi$ defines a cocycle in $\CEcoh(\g_{\R}, \K_{\R}) $ of total degree $2$.
\end{lemma}
\begin{proof}
One readily checks that $d \phi^{0} + d_{CE} \phi^{(1)} = 0 $ and $d_{CE} \phi^{0} = d \phi^{(1)} = 0$, which implies that $(d_{CE} + d) \phi = 0$.
\end{proof}

{\color{red} SIGNS !}
We may now use $\phi$ to define an $L_{\infty}$ central extension $\gtil_{\R}$ of $\g_{\R}$. As an $\R$-module, $ \gtil_{\R} = \K \oplus \g_{\R} $, and Taylor coefficients $l_1 = d, l_2 = [,] + \phi^{(1)}$, and $l_3 = \phi^{(0)}$. The following is immediate:

\begin{prop}
$H^{*}(\gtil_{\R}, l_1) = \ghat_{\R}$
\end{prop} 

\subsection{Vertex algebras}
\begin{enumerate}
\item generalities
\item Construction and structure of $V_{F}$ as a vertex algebra
\end{enumerate}


\subsection{(Pre)-factorization algebras}
\begin{enumerate}
\item Pre-factorization algebras
\item translation-invariant pre-fact algebras
\item holomorphically  translation-invariant pre-factorization algebras
\item relations with vertex algebras in one dimension. 
\end{enumerate}

\section{Factorization algebras from holomorphic fibrations}
\subsection{$dim(X)=1$ and vertex algebras}

\section{Factorization homology}

{\color{red}
\begin{enumerate}
\item Discuss factorization algebra with base a compact complex manifold and potentially non-trivial torus bundle
\item What is factorization homology or equivalently conformal blocks
\item Compute factorization homology for compact base and tirivial/non-trivial torus bundle
\end{enumerate}
}


\newpage

\begin{bibdiv}
    \begin{biblist}

\bib{CS}{article}{
   author={Crapo, Henry},
   author={Schmitt, William},
   title={A free subalgebra of the algebra of matroids},
   journal={European J. Combin.},
   volume={26},
   date={2005},
   number={7},
   pages={1066--1085},
}

\bib{CLS}{article}{
   author={Chu, Chenghao},
   author={Lorscheid, Oliver},
   author={Santhanam, Rekha},
   title={Sheaves and $K$-theory for $\Bbb F_1$-schemes},
   journal={Adv. Math.},
   volume={229},
   date={2012},
   number={4},
   pages={2239--2286},
}

\bib{CGM}{article}{
author={Crowley, C.}, 
author={Giansiracusa, N.},
author={Mundinger, J.},
title={A module-theoretic approach to matroids},
journal={preprint},
date={2017},
eprint={arXiv:1712.03440},
}

\bib{Dei}{article}{
   author={Deitmar, Anton},
   title={Remarks on zeta functions and $K$-theory over ${\bf F}_1$},
   journal={Proc. Japan Acad. Ser. A Math. Sci.},
   volume={82},
   date={2006},
   number={8},
   pages={141--146},
}

\bib{D}{article}{
author={Dyckerhoff, Tobias},
title={Higher categorical aspects of Hall algebras},
journal={preprint},
%date={2015},
eprint= {arXiv: 1505.06940},
}

\bib{DK}{article}{
author={Dyckerhoff, Tobias},
author={Kapranov, Mikhail},
title={Higher Segal Spaces I},
journal={preprint},
%date={2012},
eprint={arXiv: 1212.3563 },
}

\bib{EJS}{article}{
author={Eppolito, Chris},
author={Jun, Jaiung},
author={Szczesny, M},
title={Hopf algebras for matroids over hyperfields},
journal={preprint},
%date={2017},
eprint={arXiv:1712.08903},
}

\bib{GKT3}{article}{
author={Galvez, Imma},
author={Kock, Joachim},
author={Tonks, Andrew},
title={Decomposition spaces, incidence algebras and M�bius inversion III: the decomposition space of M�bius intervals},
journal={Adv. Math., to appear},
eprint={arXiv:1512.07580},
}

\bib{GKT2}{article}{
author={Galvez, Imma},
author={Kock, Joachim},
author={Tonks, Andrew},
title={Decomposition spaces, incidence algebras and M�bius inversion II: completeness, length filtration, and finiteness},
journal={Adv. Math., to appear},
eprint={arXiv:1512.07577},
}

\bib{GKT1}{article}{
author={Galvez, Imma},
author={Kock, Joachim},
author={Tonks, Andrew},
title={Decomposition spaces, incidence algebras and M�bius inversion I: basic theory},
journal={Adv. Math., to appear},
eprint={arXiv:1512.07573},
}

\bib{Hek}{thesis}{
author={Hekking, J.},
title={Segal Objects in Homotopical Categories \& K-theory of Proto-exact Categories},
type={Master's Thesis, Univ. of Utrecht},
date={2017},
eprint={https://www.universiteitleiden.nl/binaries/content/assets/science/mi/scripties/master/hekking_master.pdf},
}



\end{biblist}
 \end{bibdiv}


\address{\tiny DEPARTMENT OF MATHEMATICS AND STATISTICS, BOSTON UNIVERSITY, 111 CUMMINGTON MALL, BOSTON} \\
\indent \footnotesize{\email{szczesny@math.bu.edu}}

\end{document}
