\documentclass[11pt]{amsart}

\usepackage{macros-brian}

%\linespread{1.2} %for editing

\addbibresource{references.bib}

\begin{document}

\title{Chiral heterotic models}

%\author{Brian R. Williams}
%\address{School of Mathematics \\ University of Edinburgh \\ Edinburgh EH9 3FD \\ Scotland}
%\email{brian.williams@ed.ac.uk}

\begin{abstract}
\end{abstract}

\maketitle

%\vfill\eject

\setcounter{tocdepth}{1}
%\tableofcontents

\section{Definitions}

A physical heterotic $\sigma$-model is a $\sigma$-model whose source is a spin Riemann surface $(\Sigma, \sigma)$ and whose target is a K\"ahler manifold $M$ equipped with a Hermitian vector bundle $E$. 

A chiral heterotic $\sigma$-model is a holomorphic $\sigma$-model whose source is a spin Riemann surface $(\Sigma, \sigma)$ and whose target is a complex manifold $X$ equipped with a holomorphic vector bundle $E$. 

A generalized chiral heterotic $\sigma$-model is a holomorphic $\sigma$-model whose source is a spin Riemann surface $(\Sigma, \sigma)$ and whose target is a holomorphic symplectic complex supermanifold $\XX$.
 
A split generalized chiral heterotic $\sigma$-model is a chiral heterotic $\sigma$-model whose target is the cotangent bundle of some complex supermanifold. 
In this case, we can get rid of the dependence of the spin structure on the source Riemann surface by a field redefinition. 
A generalized split chiral heterotic $\sigma$-model for $\XX = \T^*{\rm Tot}(E)$ is equivalent to the chiral heterotic model. 

Some examples:
\begin{itemize}
\item 
The (curved) $\beta\gamma$ system where $\gamma$ is a holomorphic map
\[
\gamma \colon (\Sigma, \omega) \to M 
\]
is a split heterotic $\sigma$-model with target $\T^*M$. 
\item 
The large volume limit of the ordinary (Riemannian) $\sigma$-model with target a K\"ahler manifold $M$ is the sum of a split chiral heterotic $\sigma$-model with target $\T^* M$ plus its complex conjugate. 
\end{itemize}

Classically, a chiral heterotic $\sigma$-model has a symmetry induced by holomorphic reparametrization invariance. 

\begin{itemize}
\item 
The (bosonic) ambitwistor string is obtained by coupling the ghost system of worldsheet holomorphic diffeomorphisms to a chiral heterotic $\sigma$-model whose target is a symplectic reduction of $\T^* M$ where $M$ is some complex Riemannian manifold.
\end{itemize}

\section{Sheaves of chiral differential operators}

\section{Large volume limits}

As an example start with the ordinary 2d $\cN=(2,0)$ supersymmetric $\sigma$-model into a K\"ahler target $M$. 
This is a physical heterotic $\sigma$-model. 
The large volume limit decomposes as a sum of a chiral heterotic model plus an anti-chiral one. 
The chiral one is associated to the pair $(X, E) = (M, \T_M)$. 
The anti-chiral one is associated to the pair $(X,E) = (M,0)$. 

\section{Supersymmetric twists of heterotic $\sigma$-models} 

\section{BRST reductions of chiral models}

\subsection{An ADHM presentation of CDOs} 

\subsection{Chiral heterotic strings}

Perform the analysis I did with Kevin but for a general Calabi--Yau five-fold. 

\section{Ambitwistors}

These are generalized chiral models as introduced above. 

\printbibliography

\end{document}
